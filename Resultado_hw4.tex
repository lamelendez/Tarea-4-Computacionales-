\documentclass{article}
\usepackage[utf8]{inputenc}

\title{Tarea Métodos Computacionales 4}
\author{Laura Rocio Melénde Bottia }
\date{Noviembre 19 2018}

\usepackage{natbib}
\usepackage{graphicx}

\begin{document}

\maketitle

\section{Proyectil}
Elegimos como sistema ilustrativo un proyectil de masa m. Deseamos
conocer su posición y velocidad en todo instante.
Modelaremos el movimiento del proyectil suponiendo que existe un
rozamiento proporcional al cuadrado del módulo de la velocidad y de
sentido contrario a la misma.

\begin{figure}[h!]
\centering
\includegraphics[scale=0.5]{totvtw.jpg}
\caption{Mismo proyectil lanzado a diferentes ángulos }
\label{fig:universe}
\end{figure}

Runge Kutta 4-D, si funcionara veríamos una parabola alterada por el rozamiento. En este caso el codigo tiene un fallo por lo que no podemos visualizar lo deseado. Así mismo con los demás grados.


\begin{figure}[h!]
\centering
\includegraphics[scale=0.7]{45vtw.jpg}
\caption{Mismo proyectil lanzado a 45 grados. }
\label{fig:universe}
\end{figure}

En los demás casos deberíamos ver perfectamente la sensibilidad por el grado en el que se tira. 


\section{Difusión de calor}

\begin{figure}[h!]
\centering
\includegraphics[scale=0.7]{fi1o.jpg}
\caption{Fijo inicio }
\label{fig:universe}
\end{figure}

\begin{figure}[h!]
\centering
\includegraphics[scale=0.7]{fi2o.jpg}
\caption{Fijo mitad }
\label{fig:universe}
\end{figure}

\begin{figure}[h!]
\centering
\includegraphics[scale=0.7]{fi3o.jpg}
\caption{Fijo mitad}
\label{fig:universe}
\end{figure}

\begin{figure}[h!]
\centering
\includegraphics[scale=0.7]{fi4o.jpg}
\caption{Fijo equilibrio}
\label{fig:universe}
\end{figure}


\begin{figure}[h!]
\centering
\includegraphics[scale=0.7]{ab1o.jpg}
\caption{Abierto inicio }
\label{fig:universe}
\end{figure}

\begin{figure}[h!]
\centering
\includegraphics[scale=0.7]{ab2o.jpg}
\caption{Abierto mitad }
\label{fig:universe}
\end{figure}

\begin{figure}[h!]
\centering
\includegraphics[scale=0.7]{ab3o.jpg}
\caption{Abierto mitad}
\label{fig:universe}
\end{figure}

\begin{figure}[h!]
\centering
\includegraphics[scale=0.7]{ab4o.jpg}
\caption{Abierto equilibrio}
\label{fig:universe}
\end{figure}

\begin{figure}[h!]
\centering
\includegraphics[scale=0.7]{pe1o.jpg}
\caption{Periodica inicio }
\label{fig:universe}
\end{figure}

\begin{figure}[h!]
\centering
\includegraphics[scale=0.7]{pe2o.jpg}
\caption{Periodica mitad }
\label{fig:universe}
\end{figure}

\begin{figure}[h!]
\centering
\includegraphics[scale=0.7]{pe3o.jpg}
\caption{Periodica mitad}
\label{fig:universe}
\end{figure}

\begin{figure}[h!]
\centering
\includegraphics[scale=0.7]{pe4o.jpg}
\caption{Periodica equilibrio}
\label{fig:universe}
\end{figure}








\end{document}
